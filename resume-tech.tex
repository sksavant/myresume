\documentclass[a4paper,11pt]{article}
%-----------------------------------------------------------
\usepackage[top=0.75in, bottom=0.75in, left=0.55in, right=0.85in]{geometry}
\usepackage{graphicx}
\usepackage{url}
\usepackage[pdfborder=0 0 0]{hyperref}
\usepackage{palatino}
\usepackage{verbatim}
\fontfamily{Cambria}
\selectfont

\usepackage[T1]{fontenc}
\usepackage[ansinew]{inputenc}
% \usepackage{helvetica}
% \usepackage{array}
\usepackage{color}
\definecolor{mygrey}{gray}{0.90}
\textheight=9.75in
\raggedbottom
% \raggedright
\setlength{\tabcolsep}{0in}
\newcommand{\isep}{-2 pt}
\newcommand{\lsep}{-0.5cm}
\newcommand{\psep}{-0.6cm}
\renewcommand{\labelitemii}{$\circ$}
% Adjust margins
%\addtolength{\oddsidemargin}{-0.375in}
%\addtolength{\evensidemargin}{-0.375in}
%\addtolength{\textwidth}{1.75in}
%\addtolength{\topmargin}{0.1375in}
%\addtolength{\textheight}{1.75in}
\pagestyle{empty}
%-----------------------------------------------------------
%Custom commands
\newcommand{\resitem}[1]{\item #1 \vspace{-2pt}}
\newcommand{\resheading}[1]{{\small \colorbox{mygrey}{\begin{minipage}{0.975\textwidth}{\textbf{#1 \vphantom{p\^{E}}}}\end{minipage}}}}
\newcommand{\ressubheading}[3]{
\begin{tabular*}{6.62in}{l @{\extracolsep{\fill}} r}
	\textsc{{\textbf{#1}}} & \textsc{\textit{[#2]}} \\
\end{tabular*}\vspace{-8pt}}
% \textit{#3} &  \\
%-----------------------------------------------------------

\begin{document}
\hspace{0.5cm}\\
\hspace{0.5cm}\\
\hspace{0.5cm}\\
\hspace{0.5cm}\\
\hspace{0.5cm}\\
\hspace{0.5cm}\\ 
\hspace{0.5cm}\\
\hspace{0.5cm}\\
\hspace{0.5cm}\\
\hspace{0.5cm}\\
\hspace{0.5cm}\\[-0.2cm]

\resheading{\textbf{\large Scholastic Achievements}}\\[\lsep]
\begin{itemize}
  \item \textbf{All India Rank 61} in IIT-JEE (Joint Entrance Examination) - 2010. of 0.455 million students\\[-0.7cm]
  \item \textbf{All India Rank 3} in NEST (National Entrance Screening Test)-2010. of 18000 students \\[-0.7cm]
  %\item \textbf{All India Rank 168} in AIEEE (All India Engineering Entrance Examination)- 2010 of 1.1 million students

 \item Qualified to appear for the Indian National Chemistry Olympiad (\textbf{INChO}) -2010 based on \\performance in National Standard Examination in Chemistry(\textbf{NSEC}) (For \textbf{top 300} in NSEC) and has been awarded a book prize for \textbf{top 1\%} in the nation .\\[-0.7cm]
  \item Qualified to appear for the Indian National Physics Olympiad (\textbf{INPhO}) -2010 based on \\ performance in National Standard Examination in Chemistry(\textbf{NSEP}). (For \textbf{top 300} in NSEP) 
  
  \item Awarded \textbf{Certificate of Merit} by Central Board of Secondary Education (\textbf{CBSE}) for being among \textbf{top 0.1 \%} in 'Science' and 'Social Science' in All India Secondary School Examination - 2008. \\[-0.7cm]
 %\item Awarded '\textbf{Certificate of Excellence}' for securing highest aggregrate marks in the \\ school and  the title '\textbf{Amul Vidya Shree}' for Outstanding Academic performance in AISSE - 2008.

   \item Secured \textbf{AIR 4} in NIMO (National Interactive Maths Olympiad)-2009 and \textbf{AIR 5} in NISO\\ (National Interactive Science Olympiad)-2009 conducted by Eduheal Foundation.\\[-0.8cm]
  \item Secured \textbf{AIR 5} in the  XXXIX National Mathematics Talent Competition (\textbf{NMTC})-2007 conducted by Association of Mathematics Teachers of India (\textbf{AMTI}).\\[-0.8cm]
  %\item Secured \textbf{AIR 5} in FIITJEE Talent Reward Exam (FTRE) and was awarded medal for zonal topper in Mathematics, Physics and Overall.
  \item Secured \textbf{AIR 15} in 10th National Science Olympiad (NSO) - 2007 conducted by \\ Science Olympiad Foundation(SOF).
  %\item Secured \textbf{State Rank 9} in 7th National Cyber Olympiad (NCO)- 2007 conducted by SOF.
 %\item Secured \textbf{State Rank 13} in XX State Talent Search Examination- 2007 conducted by \\ Dr.A.S. Rao Awards Council and was awarded a book prize.
\end{itemize}

\resheading{\textbf{\large Summer Project - GSoC'12}} \\

\hspace{1mm}Gnucap plugin for schematic files (\textbf{Google summer of Code}) \emph{May 2012 to August 2012} \\[-0.6cm]
    \begin{itemize}
        \item Worked with the organisation 'The GNU Project' on the project 'Gnucap' (GNU Circuit Analysis Package) under the mentorship of Albert Davis \ \ \ \  [\url{gnucap.org}] \\[-0.6cm]
        \item Programmed in C++. The plugin can be loaded dynamically at run-time
        \item My work involved the following :
        \begin{itemize} \itemsep \isep
        \item Reading through the existing codebase to understand it 
        \item Figuring a way to map schematic file to Verilog-AMS and vice-versa
        \item Implementing a plugin which will import a schematic file and convert into a Verilog-AMS netlist and also export an existing  circuit in schematic format
        
    \end{itemize}
    \end{itemize}


\resheading{\textbf{\large Course Projects}}\\[\lsep]
\begin{itemize}
\item \textbf{Traveling Message Display} \emph{(Guided by Prof. M.B.Patil and J.John, EE214- Spring 2012)}\\[-0.6cm]
    \begin{itemize}\itemsep \isep
        \item Worked in a team of 3 
        \item Display a scrolling message taken using keypad on an LED Array
        \item Used an FPGA board: DE0 NANO and programmed using Verilog-HDL
        \item My work involved writing verilog modules for taking input from the keypad and processing
    \end{itemize}
    	\item \textbf{Simulation of Micromouse} \emph{(Guided by Prof.Deepak B. Phatak, CS101 - Autumn 2010)}\\[-0.6cm]
	\begin{itemize}\itemsep \isep 
	    \item Led the team of 12 members.with 3 subteams of 4 members each
	    \item Designed n$\times$n mazes, Solved them for the shortest path using Bellman-ford algorithm in C++ and Simulated the solution using EzWindows GUI.
	    \item My work involved programming the display over GUI and interlinking the different parts
	\end{itemize} 
\item \textbf{Term paper on Working of a Cordless Telephone} \emph{(Guided by Prof. Vasi J. , EE112 - Spring 2011)\\[-0.6cm]}
	\begin{itemize} \itemsep \isep
	  \item Opened and Analyzed a Cordless phone.
	  \item Worked in a team of 3 members.
	  \item Written a 12-page Term paper with details of working of the phone.
	\end{itemize}
\end{itemize}

\resheading{\textbf{\large Interests and Hobbies}} \\[\lsep] 
\begin{itemize} \itemsep \isep
    \item Digital Electronics, Microprocessors and VLSI Design
    \item Computer Science fields like Artificial Intelligence and Natural Language Processing
    \item App Development and Programming
\end{itemize}

\resheading{\textbf{\large Technical Skills}}\\[\lsep]
\begin{itemize}
  %\item Awarded Certificate of Participation in the Winter Workshop on Technical Skills conducted by \\ \ \  \ \ \ \  STUDe Club, IIT Bombay in January 2011. \\[-0.6cm]
   \item \textbf{Programming Languages}: C++,Java,Python,Ruby \  \ \ \ \ \ \ \  \textbf{Operating Systems}: Ubuntu, Windows\\[-0.6cm]
  \item \textbf{Tools}: Matlab, Mathematica, Scilab, Latex, Photoshop\ \ \ \  \textbf{Web designing}: HTML, CSS, Javascript\\[-0.6cm]
  \item \textbf{EE tools}: Spice, Verilog-HDL, Verilog-AMS , Arduino
\end{itemize}

\resheading{\textbf{\large Technical Activities}}\\[\lsep]
\begin{itemize}
 \begin{comment}
 \item Trackmania-2010
 	\begin{itemize}
 		\item Built a remote-controlled four-wheeled car (bot). 
 		\item Designed the circuit and soldered it. 
 	\end{itemize}
 \item Participated in Line-follower competition-2011.
 	\begin{itemize}
 		\item Designed and built a line-following bot.
 		\item Used IR sensors and Coded the microcontroller using Arduino software.
	\end{itemize}
 \end{comment}
 \item Yahoo! HackU -2012: Built an android app and web interface, 'MapIt' which can be used to create customizable maps of localities with greater information\\[-0.7cm]
 \item Line-follower competition-2011 : Designed and built a line-following bot using IR sensors and coding the microcontroller using Arduino software\\[-0.7cm]
\item Trackmania-2010 : Built a remote-controlled four-wheeled car (bot).
\end{itemize}

\resheading{\textbf{\large Extra Curricular Activities and Achievements}}\\[\lsep]
%\resheading{EXTRA CURRICULAR ACTIVITIES AND ACHIEVEMENTS}\\[\lsep]
\begin{itemize}\itemsep \isep
  \item Participated in \textbf{Unnati}, the \textbf{NSS} (National Service Scheme) group of IIT Bombay.
	\begin{itemize} 
	 \item Has been involved with the \textbf{GRA} (Group for Rural Activities) as part of curriculum in I year\\[-0.6cm]
	 \item Went to Village trips in Autumn 2010 and Spring 2011.\\[-0.6cm]
	 \item Continued as a voluntary member of the NSS Team in the subsequent year.
	  \end{itemize}
  \item Worked as 'Organiser' in \textbf{Techfest-2011} in the Lecture Series department. \\[-0.6cm]
  %\item Currently working as '\textbf{Coordinator}' in Techfest-2012.
  \item Participated in the Inter-hostel Hockey GC. 
\end{itemize}
	

\resheading{\textbf{\large Courses currently taking (Autumn 2012)}}\\[\lsep]
 % \begin{itemize}
  \begin{comment}
  \item \textbf{Courses Taken}	
	\begin{itemize}\itemsep \isep
	  \item Calculus, Linear Algebra, Differential Equations.
	  \item Electricity and Magnetism, Chemistry. \\
	  \item Data Analysis and Interpretation, Computer Programming and Utilization.
	  \item Introduction to Electrical Systems, Introduction to Electronics 
	  \item Workshop Practice, Engineering Drawing, Physics Lab, Chemistry Lab. \\
	  \item Complex Analysis, Differential Equations, Economics 
	  \item Network Theory, Electronic Devices and Circuits. \\
	  \item Discrete Structures, A First Course in Optimization.
	  \item Experimental and Measurement Laboratory, Electronic Devices Lab.
	\end{itemize}
	\end{comment}
% \item \textbf{Courses currently taking this semester}
 	\begin{itemize} \itemsep \isep
 	    \item EE Core Courses: \\[-0.6cm]
 	    \begin{itemize}\itemsep \isep
 	        \item Microprocessors, Microprocessors Lab
 	        \item Communication Systems, Communications Lab
 	        \item Electromagnetic Waves
 	    \end{itemize}
 	    \item Additional Courses: \\[-0.6cm]
 	    \begin{itemize}\itemsep \isep
 	        \item Foundations of VLSI CAD
 	        \item Artificial Intelligence
 	        \item Data Structures and Algorithms
 	    \end{itemize}
 	    \item Institute Core courses: \\[-0.6cm]
 	    \begin{itemize}\itemsep \isep
 	        \item Psychology
 	    \end{itemize}
 	\end{itemize}
%\end{itemize}


\end{document}
