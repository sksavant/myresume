\documentclass[a4paper,11pt]{article}
%-----------------------------------------------------------
\usepackage[top=0.75in, bottom=0.75in, left=0.55in, right=0.85in]{geometry}
\usepackage{graphicx}
\usepackage{url}
\usepackage[pdfborder=0 0 0]{hyperref}
\usepackage{palatino}
\usepackage{verbatim}
\usepackage{array}
\fontfamily{Cambria}
\selectfont

\usepackage[T1]{fontenc}
\usepackage[ansinew]{inputenc}
% \usepackage{helvetica}
% \usepackage{array}
\usepackage{color}
\definecolor{mygrey}{gray}{0.90}
\textheight=9.75in
\raggedbottom
% \raggedright
\setlength{\tabcolsep}{0in}
\newcommand{\isep}{-2 pt}
\newcommand{\lsep}{-0.5cm}
\newcommand{\psep}{-0.6cm}
\renewcommand{\labelitemii}{$\circ$}
% Adjust margins
%\addtolength{\oddsidemargin}{-0.375in}
%\addtolength{\evensidemargin}{-0.375in}
%\addtolength{\textwidth}{1.75in}
%\addtolength{\topmargin}{0.1375in}
%\addtolength{\textheight}{1.75in}
\pagestyle{empty}
%-----------------------------------------------------------
%Custom commands
\newcommand{\resitem}[1]{\item #1 \vspace{-2pt}}
\newcommand{\resheading}[1]{{\small \colorbox{mygrey}{\begin{minipage}{0.975\textwidth}{\textbf{#1 \vphantom{p\^{E}}}}\end{minipage}}}}
\newcommand{\ressubheading}[3]{
\begin{tabular*}{6.62in}{l @{\extracolsep{\fill}} r}
    \textsc{{\textbf{#1}}} & \textsc{\textit{[#2]}} \\
\end{tabular*}\vspace{-8pt}}
% \textit{#3} &  \\
%-----------------------------------------------------------

\begin{document}

\hspace{0.5cm}\\
\hspace{0.5cm}\\
\hspace{0.5cm}\\
\hspace{0.5cm}\\
\hspace{0.5cm}\\
\hspace{0.5cm}\\
\hspace{0.5cm}\\
\hspace{0.5cm}\\
\hspace{0.5cm}\\
\hspace{0.5cm}\\
\hspace{0.5cm}\\[-0.2cm]

\begin{comment}
\begin{tabular}{m{8cm} m{8cm}} 
    \large{KRISHNA SAVANT SYREDDY}  & 100070056\\
    \large{Electrical Engineering}  & UG Third Year (B.Tech)\\
    \large{Indian Institute of Technology, Bombay} & DOB: 19-06-1993 \\
    \large{\url{sk.savant@iitb.ac.in}} & +919757033532
\end{tabular}

\resheading{\textbf{\large{Education}}}
\begin{itemize}
    \item Undergraduate Degree - IIT Bombay - 2008 to present  - 8.59
    \item Intermediate/+2 - BIE, AP - SR Junior College - 2010 - 90.50
    \item Matriculation - CBSE - Warangal Public School - 2008 - 93.00
\end{itemize}
\end{comment}


\resheading{\textbf{\large Research Interests}} \\[\lsep] 
\begin{itemize} \itemsep \isep
    \item Robotics (Planning and Localization), Artificial Intelligence 
    \item VLSI Design, High Level Synthesis, VLSI CAD
   % \item App Development and Programming
\end{itemize}


\resheading{\textbf{\large{Current Projects}}}
\begin{itemize}

    \item \textbf{BTP}

    \item \textbf{Analyzing Point Clouds - Bridge Inspection Project} \hfill (\emph{May 2013 to July 2013}) \\
        Field Robotics Center, Robotics Institute, Carnegie Mellon University 

    \item \textbf{AUVSI Robosub 2013, San Deigo, CA} \hfill \emph{(September 2012 to May 2013) } \\
        Designing and developing an unmanned autonomous underwater vehicle (AUV) that localizes itself and performs realistic missions based on feedback from visual, inertial, acoustic and pressure sensors using thrusters/propellers. \hfill  [\url{http://auv-iitb.org}] \\ 
        \emph{(Guided by Prof. Hemendra Arya and Prof. Leena Vachhani)} \\[-0.6cm]
        \begin{itemize}
            %\item Working on  towards competing in AUVSI Robosub 2013 
            \item Working on the navigation system consisting of planning, localizing and accurate maneuvring of the vehicle.
            \item Developing algorithms for fusing and filtering data from various sensors and control the vehicle. Working on ROS (Robot Operating System), an opensource software framework providing hardware abstraction and Inter Process Communication mechanism.
        \end{itemize}
\begin{comment} %Ditched courses 318 and 739 worked on robosub 
\item \textbf{Robotic Sniffer Dog} \hfill \emph{(Guided by Prof. J. John and  D. Sharma, EE318 - Spring 2013)}
        \begin{itemize}
            \item Building a working prototype of a robotic control system which can be controlled from a considerable distance (using repeaters) through WLAN, Zigbee protocol relying on visual data from cameras.
            \item Consists of a sensor board which can detect explosive vapors in parts per billion. 
        \end{itemize}
    \item \textbf{Processor design on FPGA}  \hfill \emph{(Guided by Prof. V. Singh, EE739 - Spring 2013)}
        \begin{itemize}
            \item Involves designing a specific processor architecture on FPGA
            \item Literature survey about available processor architectures, proposing a better one and demonstrating using FPGA
        \end{itemize}
    %\item \textbf{ Speech recognition - Artificial Intelligence} \emph{(Guided by Prof. G. Sivakumar, CS621 - Autumn 2012)}
\end{comment}

\end{itemize}

\resheading{\textbf{\large Google Summer of Code - GSoC'12}} \\

\hspace{1mm} Worked on a FOSS project \textbf{Gnucap plugin for schematic files} \hfill \emph{(May 2012 to August 2012)} \\[-0.6cm]
    \begin{itemize}
        \item Worked with the organisation 'GNU Project' on the project 'Gnucap' (GNU Circuit Analysis Package) under the mentorship of Albert Davis. \hfill  [\url{http://gnucap.org}] \\[-0.6cm]
%%        \item My work involved the following :
 %       \begin{itemize} \itemsep \isep
  %      %\item Reading through the existing codebase to understand it 
   %     \item Figuring a way to map schematic file to Verilog-AMS and vice-versa
        \item Implemented a schematic parser which provides interchange of data between simulatable Verilog-AMS netlist and gEDA/gschem schematic format.
   % \end{itemize}
    \end{itemize}


\resheading{\textbf{\large Key Academic Projects}}\\[\lsep]
\begin{itemize}
    \item \textbf{ epsilon-to-verilog: An Educational Hardware Compiler} \hfill \emph{(Guided by Prof. S. Patkar, Sep-Nov, 2012) }  \\[-0.6cm]
    \begin{itemize} \itemsep \isep
        \item epsilon-to-verilog synthesizes programs written in a new custom minimalistic high level language epsilon to hardware description languages
        \item The tool parses the cfg (control flow graph) generated by epsilon and does scheduling and allocation to generate hardware description in verilog.
   \end{itemize}
    \item \textbf{Technology Mapping - VLSI CAD} \hfill \emph{(Guided by Prof. S.Patkar, EE677 - Autumn 2012)} \\[-0.6cm]
    \begin{itemize} \itemsep \isep
        \item Modeling the problem of technology mapping as a tree covering problem using pattern trees of the library gates.
        \item Implementing using python graph-tool library
    \end{itemize}
%\pagebreak

    \item \textbf{Traveling Message Display} \hfill \emph{(Guided by Prof. M.B.Patil and J.John, EE214- Spring 2012)}\\[-0.6cm]
    \begin{itemize}\itemsep \isep
        \item Worked in a team of 3 members
        \item Display a scrolling message taken using keypad on an LED Array
        \item Used an FPGA board: DE0 NANO and programmed using Verilog-HDL
        \item My work involved writing verilog modules for taking input from the keypad and processing
    \end{itemize}
        \item \textbf{Simulation of Micromouse} \hfill \emph{(Guided by Prof.Deepak B. Phatak, CS101 - Autumn 2010)}\\[-0.6cm]
    \begin{itemize}\itemsep \isep 
        \item Led the team of 12 members with 3 subteams of 4 members each
        \item Designed n$\times$n mazes, Solved them for the shortest path using Bellman-ford algorithm in C++ and Simulated the solution using EzWindows GUI.
        \item My work involved programming the display over GUI and interlinking the different parts
    \end{itemize}
\begin{comment}
\item \textbf{Term paper on Working of a Cordless Telephone} \emph{(Guided by Prof. Vasi J. , EE112 - Spring 2011)\\[-0.6cm]}
    \begin{itemize} \itemsep \isep
      \item Opened and Analyzed a Cordless phone.
      \item Worked in a team of 3 members.
      \item Written a 12-page Term paper with details of working of the phone.
    \end{itemize}
\end{comment}
\end{itemize}


\resheading{\textbf{\large Scholastic Achievements}}\\[\lsep]
\begin{itemize}
  \item \textbf{All India Rank 61} in IIT-JEE (Joint Entrance Examination) - 2010 of 0.455 million students\\[-0.7cm]
  %\item \textbf{All India Rank 3} in NEST (National Entrance Screening Test)-2010 of 18000 students \\[-0.7cm]
  %\item \textbf{All India Rank 168} in AIEEE (All India Engineering Entrance Examination)- 2010 of 1.1 million students

  %\item Qualified to appear for the Indian National Chemistry Olympiad (\textbf{INChO}) -2010 based on \\performance in National Standard Examination in Chemistry(\textbf{NSEC}) (For \textbf{top 300} in NSEC) and has been awarded a book prize for \textbf{top 1\%} in the nation .\\[-0.7cm]
  %\item Qualified to appear for the Indian National Physics Olympiad (\textbf{INPhO}) -2010 based on \\ performance in National Standard Examination in Chemistry(\textbf{NSEP}). (For \textbf{top 300} in NSEP) 

  \item Awarded \textbf{Certificate of Merit} by Central Board of Secondary Education (\textbf{CBSE}) for being among \textbf{top 0.1 \%} in 'Science' and 'Social Science' in All India Secondary School Examination - 2008. \\[-0.7cm]
  %\item Awarded '\textbf{Certificate of Excellence}' for securing highest aggregrate marks in the \\ school and  the title '\textbf{Amul Vidya Shree}' for Outstanding Academic performance in AISSE - 2008.

  %\item Secured \textbf{AIR 4} in NIMO (National Interactive Maths Olympiad)-2009 and \textbf{AIR 5} in NISO\\ (National Interactive Science Olympiad)-2009 conducted by Eduheal Foundation.\\[-0.8cm]
  %\item Secured \textbf{AIR 5} in the  XXXIX National Mathematics Talent Competition (\textbf{NMTC})-2007 conducted by Association of Mathematics Teachers of India (\textbf{AMTI}).\\[-0.8cm]
  %\item Secured \textbf{AIR 5} in FIITJEE Talent Reward Exam (FTRE) and was awarded medal for zonal topper in Mathematics, Physics and Overall.
  \item Secured \textbf{All India Rank 15} in 10th National Science Olympiad (NSO) - 2007 conducted by Science Olympiad Foundation(SOF).
  %\item Secured \textbf{State Rank 9} in 7th National Cyber Olympiad (NCO)- 2007 conducted by SOF.
  %\item Secured \textbf{State Rank 13} in XX State Talent Search Examination- 2007 conducted by \\ Dr.A.S. Rao Awards Council and was awarded a book prize.
\end{itemize}

\resheading{\textbf{\large Technical Skills}}\\[\lsep]
\begin{itemize}
  %\item Awarded Certificate of Participation in the Winter Workshop on Technical Skills conducted by \\ \ \  \ \ \ \  STUDe Club, IIT Bombay in January 2011. \\[-0.6cm]
   \item \textbf{Programming Languages}: C++,Java,Python,Ruby \hfill \textbf{Operating Systems}: Linux, Windows\\[-0.6cm]
  \item \textbf{Tools}: Latex, Scilab, Mathematica, Photoshop\hfill \textbf{Web development}: HTML, CSS, JS, Django\\[-0.6cm]
  \item \textbf{EE tools}: ngspice, gnucap,  gEDA tools, Eagle, Verilog-HDL, Verilog-AMS, Icarus verilog \\ \hspace*{1.5cm} Bluespec System Verilog (BSV), Modelsim, Altera Quartus
\end{itemize}

\begin{comment}
\resheading{\textbf{\large Misc. Technical Activities}}\\[\lsep]
\begin{itemize}
 \item Trackmania-2010
    \begin{itemize}
        \item Built a remote-controlled four-wheeled car (bot). 
        \item Designed the circuit and soldered it. 
    \end{itemize}
 \item Participated in Line-follower competition-2011.
    \begin{itemize}
        \item Designed and built a line-following bot.
        \item Used IR sensors and Coded the microcontroller using Arduino software.
    \end{itemize}
 \item Yahoo! HackU -2012: Built an android app and web interface, 'MapIt' which can be used to create customizable maps of localities with greater information\\[-0.7cm]
 \item Line-follower competition-2011 : Designed and built a line-following bot using IR sensors and coding the microcontroller using Arduino software\\[-0.7cm]
\item Trackmania-2010 : Built a remote-controlled differential drive land bot.
\end{itemize}
\end{comment}

\resheading{\textbf{\large Extra Curricular Activities}}\\[\lsep]
%\resheading{EXTRA CURRICULAR ACTIVITIES AND ACHIEVEMENTS}\\[\lsep]
\begin{itemize}\itemsep \isep
        \item Participated in \textbf{Unnati}, the \textbf{NSS} (National Service Scheme) group of IIT Bombay. \\[-0.6cm]
    \begin{itemize} 
     \item Has been involved with the \textbf{GRA} (Group for Rural Activities) as part of curriculum in I year\\[-0.6cm]
     \item Went to Village trips in Autumn 2010 and Spring 2011.\\[-0.6cm]
     \item Continued as a voluntary member of the NSS Team in the subsequent year.
      \end{itemize}
  %\item Currently working as '\textbf{Coordinator}' in Techfest-2012.
  \item Participated in the Inter-hostel Hockey GC. 
\end{itemize}
    

\resheading{\textbf{\large Additional Courses taken / currently taking}}\\[\lsep]
 % \begin{itemize}
  \begin{comment}
  \item \textbf{Courses Taken}  
    \begin{itemize}\itemsep \isep
      \item Calculus, Linear Algebra, Differential Equations.
      \item Electricity and Magnetism, Chemistry. \\
      \item Data Analysis and Interpretation, Computer Programming and Utilization.
      \item Introduction to Electrical Systems, Introduction to Electronics 
      \item Workshop Practice, Engineering Drawing, Physics Lab, Chemistry Lab. \\
      \item Complex Analysis, Differential Equations, Economics 
      \item Network Theory, Electronic Devices and Circuits. \\
      \item Discrete Structures, A First Course in Optimization.
      \item Experimental and Measurement Laboratory, Electronic Devices Lab.
    \end{itemize}
    \end{comment}
% \item \textbf{Courses currently taking this semester}
\begin{comment}
    \begin{itemize} \itemsep \isep
        \item EE Core Courses: \\[-0.6cm]
        \begin{itemize}\itemsep \isep
            \item Microprocessors, Microprocessors Lab
            \item Communication Systems, Communications Lab
            \item Electromagnetic Waves
        \end{itemize}
        \item Additional Courses: \\[-0.6cm]
        \begin{itemize}\itemsep \isep
            \item Foundations of VLSI CAD
            \item Artificial Intelligence
            \item Data Structures and Algorithms
        \end{itemize}
        \item Institute Core courses: \\[-0.6cm]
        \begin{itemize}\itemsep \isep
            \item Psychology
        \end{itemize}
    \end{itemize}
\end{comment}
%\end{itemize}

\begin{tabbing}\itemsep \isep
        \hspace{1cm}\= \emph{Electronic Design Lab} \hspace{3cm} \quad\= \emph{Processor Design} \\
                \> \emph{Computer Networks}  \>Foundations of VLSI CAD \\
                \> Artificial Intelligence  \> Data Structures and Algorithms \\
                \> Discrete Structures \> First Course in Optimisation \\
                \>Introduction to Quantum Mechanics \\
 \end{tabbing}

\begin{comment}
\resheading{\large{Courses}}
    {\renewcommand{\arraystretch}{1.5}
    \begin{tabular}{|l|c|l|c|}
\hline
\multicolumn{4}{|c|}{\bfseries Electrical and Electronics}\\
\hline
Course & Grade & Course & Grade \\
\hline
Introduction to Electrical Systems &8 & Introduction to Electronics &10 \\
\hline
Network theory  &8 & Electronic Devices and Circuits  &7 \\
\hline
Signals and Systems  &7 & Electrical Machines and Power Electronics &7  \\
\hline
Digital Systems  &10 & Analog Circuits  &10 \\
\hline
Electromagnetic Waves  &NA & Communication Systems  &NA \\
\hline
Microprocessors  &NA & Probability and Random Processes  &NA \\
\hline
Control Systems  &NA & Digital Signal Processing  &NA \\
\hline
Power Systems  &NA & Digital Communications  &NA \\
\hline
System Design &NA & Testing and Verification of VLSI Circuits  &NA\\
\hline
A first course in optimization  &9 & Foundation of VLSI CAD  & NA \\
\hline
\multicolumn{2}{|c}{\bfseries Computer Science} &
\multicolumn{2}{c|}{\bfseries Mathematics} \\
\hline
Computer Programming and Utilization  &10 & Calculus  &10 \\
\hline
Data Structures and Algorithms  &NA & Linear Algebra  &8 \\
\hline
Discrete Structures  &6 &  Differential Equations I  &7 \\
\hline
Artificial Intelligence  &NA & Differential Equations II  &7 \\
\hline
Computer Networks  &NA & Data Analysis and Interpretation &10 \\
\hline
\multicolumn{4}{|c|}{\bfseries Laboratories} \\
\hline
Digital Electronics Lab  &8 & Analog Circuits Lab  &8 \\
\hline
Microprocessors Lab  &NA & Electrical Machines &8 \\
\hline
Electronics Devices  &7 & Communications Lab  &NA \\
\hline
Experimentation and Measurements  &8 & Control Systems Lab  &NA  \\
\hline
Engineering Drawing  &9 & Workshop Practice   &9 \\
\hline
Chemistry Lab  &10 & Physics Lab  &8 \\
\hline
\multicolumn{4}{|c|}{\bfseries Sciences and Humanities}\\
\hline
Electricity and Magnetism &9  & Chemistry  &8 \\
\hline
Economics  &8 & Psychology  &NA \\
\hline
\end{tabular}}\\
\\
{\small The courses marked with NA shall be covered in the academic
year 2012-13}
\end{comment}

\end{document}
