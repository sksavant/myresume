\documentclass[a4paper,11pt]{article}
%-----------------------------------------------------------
\usepackage[top=0.75in, bottom=0.75in, left=0.55in, right=0.85in]{geometry}
\usepackage{graphicx}
\usepackage{url}
\usepackage{palatino}
\fontfamily{Cambria}
\selectfont

\usepackage[T1]{fontenc}
\usepackage[ansinew]{inputenc}
% \usepackage{helvetica}
% \usepackage{array}
\usepackage{color}
\definecolor{mygrey}{gray}{0.90}
\textheight=9.75in
\raggedbottom
% \raggedright
\setlength{\tabcolsep}{0in}
\newcommand{\isep}{-2 pt}
\newcommand{\lsep}{-0.5cm}
\newcommand{\psep}{-0.6cm}
\renewcommand{\labelitemii}{$\circ$}
% Adjust margins
%\addtolength{\oddsidemargin}{-0.375in}
%\addtolength{\evensidemargin}{-0.375in}
%\addtolength{\textwidth}{1.75in}
%\addtolength{\topmargin}{0.1375in}
%\addtolength{\textheight}{1.75in}
\pagestyle{empty}
%-----------------------------------------------------------
%Custom commands
\newcommand{\resitem}[1]{\item #1 \vspace{-2pt}}
\newcommand{\resheading}[1]{{\small \colorbox{mygrey}{\begin{minipage}{0.975\textwidth}{\textbf{#1 \vphantom{p\^{E}}}}\end{minipage}}}}
\newcommand{\ressubheading}[3]{
\begin{tabular*}{6.62in}{l @{\extracolsep{\fill}} r}
	\textsc{{\textbf{#1}}} & \textsc{\textit{[#2]}} \\
\end{tabular*}\vspace{-8pt}}
% \textit{#3} &  \\
%-----------------------------------------------------------

\begin{document}
\hspace{0.5cm}\\
\hspace{0.5cm}\\
\hspace{0.5cm}\\
\hspace{0.5cm}\\
\hspace{0.5cm}\\
\hspace{0.5cm}\\ 
\hspace{0.5cm}\\
\hspace{0.5cm}\\
\hspace{0.5cm}\\
\hspace{0.5cm}\\
\hspace{0.5cm}\\[-0.2cm]

\resheading{\textbf{\large Scholastic Achievements}}\\[\lsep]
\begin{itemize}
  \item \textbf{All India Rank 61} in IIT-JEE (Joint Entrance Examination) - 2010.\\[-0.7cm]
  \item \textbf{All India Rank 3} in NEST (National Entrance Screening Test)-2010.\\[-0.7cm]
  \item \textbf{All India Rank 168} in AIEEE (All India Engineering Entrance Examination)- 2010.

 \item Qualified to appear for the Indian National Chemistry Olympiad (\textbf{INChO}) -2010 based on \\performance in National Standard Examination in Chemistry(\textbf{NSEC}) and has been awarded a book prize.\\[-0.7cm]
  \item Qualified to appear for the Indian National Physics Olympiad (\textbf{INPhO}) -2010 based on \\ performance in National Standard Examination in Chemistry(\textbf{NSEP}).  
  
  \item Awarded \textbf{Certificate of Merit} by Central Board of Secondary Education (\textbf{CBSE}) for being among \textbf{top 0.1 \%} in 'Science' and 'Social Science' in All India Secondary School Examination - 2008. \\[-0.7cm]
 \item Awarded '\textbf{Certificate of Excellence}' for securing highest aggregrate marks in the \\ school and  the title '\textbf{Amul Vidya Shree}' for Outstanding Academic performance in AISSE - 2008.

   \item Secured \textbf{AIR 4} in NIMO (National Interactive Maths Olympiad)-2009 and \textbf{AIR 5} in NISO\\ (National Interactive Science Olympiad)-2009 conducted by Eduheal Foundation.\\[-0.8cm]
  \item Secured \textbf{AIR 5} in the  XXXIX National Mathematics Talent Competition (\textbf{NMTC})-2007 conducted by Association of Mathematics Teachers of India (\textbf{AMTI}).\\[-0.8cm]
  \item Secured \textbf{AIR 5} in FIITJEE Talent Reward Exam (FTRE) and was awarded medal for zonal topper in Mathematics, Physics and Overall.
  \item Secured \textbf{AIR 15} in 10th National Science Olympiad (NSO) - 2007 conducted by \\ Science Olympiad Foundation(SOF).\\[-0.8cm]
  \item Secured \textbf{State Rank 9} in 7th National Cyber Olympiad (NCO)- 2007 conducted by SOF.\\[-0.8cm]
 \item Secured \textbf{State Rank 13} in XX State Talent Search Examination- 2007 conducted by \\ Dr.A.S. Rao Awards Council and was awarded a book prize.
\end{itemize}


\resheading{\textbf{\large Course Projects}}\\[\lsep]
\begin{itemize}
\item \textbf{Simulation of Micromouse} \emph{(Guided by Prof.Deepak B. Phatak, CS101 - Autumn 2010)}\\[-0.6cm]
	\begin{itemize}\itemsep \isep 
	    \item Led the team of 12 members.
	    \item Designed n$\times$n mazes.
	    \item Solved them for the minimum path using Bellman-ford algorithm in C++. 
	    \item Simulated the solution using EzWindows GUI.
	\end{itemize} 
\item \textbf{Term paper on Working of a Cordless Telephone} \emph{(Guided by Prof. Vasi J. , EE112 - Spring 2011)\\[-0.6cm]}
	\begin{itemize} \itemsep \isep
	  \item Opened and Analyzed a Cordless phone.
	  \item Worked in a team of 3 members.
	  \item Written a 12-page Term paper with details of working of the phone.
	\end{itemize}
\end{itemize}


\resheading{\textbf{\large Extra Curricular Activities and Achievements}}\\[\lsep]
%\resheading{EXTRA CURRICULAR ACTIVITIES AND ACHIEVEMENTS}\\[\lsep]
\begin{itemize}
  \item Participated in \textbf{Unnati}, the \textbf{NSS} (National Service Scheme) group of IIT Bombay.
	\begin{itemize}
	 \item Has been involved with the \textbf{GRA} (Group for Rural Activities) as part of curriculum in First year\\[-0.6cm]
	 \item Went to Village trips in Autumn 2010 and Spring 2011.
	 \item Continuing as a voluntary member of the NSS Team in the subsequent year.
	  \end{itemize}
  \item Worked as 'Organiser' in \textbf{Techfest-2011}, Asia's largest Science and Technology festival, in the Lecture Series department. \\[-0.6cm]
  \item Currently working as '\textbf{Coordinator}' in Techfest-2012.
  \item Participated in the Inter-hostel Hockey GC. 
\end{itemize}

\resheading{\textbf{\large Technical Skills}}\\[\lsep]
\begin{itemize}
  \item Awarded Certificate of Participation in the Winter Workshop on Technical Skills conducted by \\ \ \  \ \ \ \  STUDe Club, IIT Bombay in January 2011. \\[-0.6cm]
  \item \textbf{Programming Languages}: C++,Java,Python\\[-0.6cm]
  \item \textbf{Operating Systems}: Linux-Ubuntu, Windows\\[-0.6cm]
  \item \textbf{Tools}: Matlab, Mathematica, Scilab, Latex, Photoshop\\[-0.6cm]
  \item \textbf{Web designing}: HTML, CSS, Javascript.
\end{itemize}

\resheading{\textbf{\large Technical Activities}}
\begin{itemize}
 \item Participated in Trackmania-2010 which involved building a remote-controlled four-wheeled car (bot). \\[-0.7cm]
 	
 \item Participated in Line-follower competition-2011 : Designed and built a line-following bot using IR sensors and coding the microcontroller using Arduino software.
\end{itemize}


\resheading{\textbf{\large Courses}}\\[\lsep]
  \begin{itemize}
  \item \textbf{Courses Taken}	
	\begin{itemize}\itemsep \isep
	  \item Calculus, Linear Algebra, Differential Equations.
	  \item Electricity and Magnetism, Chemistry. \\
	  \item Data Analysis and Interpretation, Computer Programming and Utilization.
	  \item Introduction to Electrical Systems, Introduction to Electronics 
	  \item Workshop Practice, Engineering Drawing, Physics Lab, Chemistry Lab. \\
	  \item Complex Analysis, Differential Equations, Economics 
	  \item Network Theory, Electronic Devices and Circuits. \\
	  \item Discrete Structures, A First Course in Optimization.
	  \item Experimental and Measurement Laboratory, Electronic Devices Lab.
	\end{itemize}
 \item \textbf{Courses currently taking this semester}
 	\begin{itemize} \itemsep \isep
 	  \item Signals and Systems, Electrical Machines and Power Electronics.
 	  \item Analog Circuits, Digital Systems.
 	  \item Analog Lab, Digital Circuits Lab, Machines Lab.
 	  \item Introduction to Quantum mechanics, Introduction to MEMS
 	\end{itemize}
\end{itemize}


\end{document}
