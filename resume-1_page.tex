\documentclass[a4paper,10pt]{article}
%-----------------------------------------------------------
\usepackage[top=0.75in, bottom=0.75in, left=0.55in, right=0.85in]{geometry}
\usepackage{graphicx}
\usepackage{url}
\usepackage{palatino}
\usepackage{verbatim}
\usepackage[pdfborder=0 0 0]{hyperref}
\fontfamily{Cambria}
\selectfont

\usepackage[T1]{fontenc}
\usepackage[ansinew]{inputenc}
% \usepackage{helvetica}
% \usepackage{array}
\usepackage{color}
\definecolor{mygrey}{gray}{0.90}
\textheight=9.75in
\raggedbottom
% \raggedright
\setlength{\tabcolsep}{0in}
\newcommand{\isep}{-2 pt}
\newcommand{\lsep}{-0.5cm}
\newcommand{\psep}{-0.6cm}
\renewcommand{\labelitemii}{$\circ$}
% Adjust margins
%\addtolength{\oddsidemargin}{-0.375in}
%\addtolength{\evensidemargin}{-0.375in}
%\addtolength{\textwidth}{1.75in}
%\addtolength{\topmargin}{0.1375in}
%\addtolength{\textheight}{1.75in}
\pagestyle{empty}
%-----------------------------------------------------------
%Custom commands
\newcommand{\resitem}[1]{\item #1 \vspace{-2pt}}
\newcommand{\resheading}[1]{{\small \colorbox{mygrey}{\begin{minipage}{0.975\textwidth}{\textbf{#1 \vphantom{p\^{E}}}}\end{minipage}}}}
\newcommand{\ressubheading}[3]{
\begin{tabular*}{6.62in}{l @{\extracolsep{\fill}} r}
	\textsc{{\textbf{#1}}} & \textsc{\textit{[#2]}} \\
\end{tabular*}\vspace{-8pt}}
% \textit{#3} &  \\
%-----------------------------------------------------------

\begin{document}
\hspace{0.5cm}\\
\hspace{0.5cm}\\
\hspace{0.5cm}\\
\hspace{0.5cm}\\
\hspace{0.5cm}\\
\hspace{0.5cm}\\ 
\hspace{0.5cm}\\
\hspace{0.5cm}\\
\hspace{0.5cm}\\[-0.2cm]


\resheading{\textbf{\large{Major Projects}}}
\begin{itemize}
    \item \textbf{Analyzing Point Clouds - Bridge Inspection Project} \hfill (\emph{Dr. Sebastian Scherer, May 2013 to July 2013}) \\
        \emph{(Visiting Summer Scholar at Field Robotics Center, Robotics Institute, Carnegie Mellon University)} \\[-0.6cm]
        \begin{itemize} \itemsep \isep
            \item Implemented algorithms to build a 3D model of bridge from laser scan data obtained from an UAV
            \item Developed techniques to analyze coverage of bridge from different viewpoints
            \item Presented poster at UG Research Symposium 2013
        \end{itemize}

    \item \textbf{Autonomous Underwater Vehicle Project (AUV-IITB)} \hfill \emph{(September 2012 to Present) } \\[-0.6cm]
    \begin{itemize} \itemsep \isep
        \item Designed and developed an unmanned autonomous underwater vehicle (AUV) that localizes itself and performs realistic missions based on feedback form visual, inertial, acoustic and depth sensors. 
        \item Worked on navigation system : planning, localization and accurate maneuvring and developed algorithms for fusing and filtering data from various sensors and control vehicle. \\[-0.6cm]
    \end{itemize}
    \begin{comment}
    \item \textbf{ High Level Synthesis using Legup} \hfill \emph{(Guided by Prof. S. Patkar, September 2012 to Present) }  \\[-0.6cm]
    \begin{itemize} \itemsep \isep 
        \item Working under High Power Computing Lab, EE Dept. on HLS to implement a memory model using stacks or queues to improve the synthesis of specific applications with regular and iterative 
        \item Building up on legup infrastructure to analyze different high level synthesis techniques
        \end{itemize}
    \end{comment}


%\resheading{\textbf{\large Google Summer of Code Project} \\ 
    \item \textbf {Google Summer of Code Project} \hfill  \emph{(May 2012 to August 2012)} \\[-0.6cm]
%\hspace{1mm} Worked on a FOSS project \textbf{Gnucap plugin for schematic files} \hfill \emph{(May 2012 to August 2012)} \\[-0.6cm]
    \begin{itemize}
        \item Worked under organisation 'The GNU Project' on the project Gnucap (GNU Circuit Analysis Package)\\[-0.6cm]
        \item Worked on a gnucap language plugin for schematic files \\[-0.6cm]
        \item Implemented a plugin to convert a schematic of a circuit into Verilog-AMS netlist and vice-versa.\\[-0.5cm]
       % \item Programmed in C++ and object-oriented programming paradigms.\\[-0.5cm]
    \end{itemize}
\end{itemize}

\resheading{\textbf{\large Academic Projects}}
\begin{itemize}
    \item \textbf{ epsilon-to-verilog: An Educational Hardware Compiler} \hfill \emph{(Guided by Prof. S. Patkar, Sep-Nov, 2012) }  \\[-0.6cm]
        \begin{itemize} \itemsep \isep
        \item Developed a custom minimalistic high level language epsilon using python
        \item Implemented scheduling and allocation to generate hardware description from control flow graph
   \end{itemize}
\item \textbf{Technology Mapping - VLSI CAD} \hfill  \hfill \emph{(Guided by Prof. S.Patkar, EE677 - Autumn 2012)} \\[-0.6cm]
    \begin{itemize} 
        \item Modeling the problem of technology mapping as a tree covering problem using pattern trees. \\[-0.6cm]
        \item Implementing using python graph-tool library \\[-0.6cm]
    \end{itemize}
\item \textbf{Traveling Message Display} \hfill \emph{(Guided by Prof. M.B.Patil and J.John, EE214 - Spring 2012)}\\[-0.6cm]
    \begin{itemize}
        \item Worked in a team of 3 to display a scrolling message on an LED Array using FPGA DE0 NANO\\[-0.6cm]
        \item My work involved writing verilog modules for taking input from the keypad and processing it.\\[-0.6cm]
    \end{itemize}
\item \textbf{Simulation of Micromouse} \hfill \emph{(Guided by Prof.Deepak B. Phatak, CS101 - Autumn 2010)}\\[-0.7cm]
	\begin{itemize}
	    \item Led the team of 12 members in designing and solved n$\times$n mazes them for the minimum path using Bellman-ford algorithm in C++ and simulated using EzWindows GUI.\\[-0.6cm]
	\end{itemize} 
\begin{comment}
\item \textbf{Term paper on Working of a Cordless Telephone} \emph{(Guided by Prof. Vasi J. , EE112 - Spring 2011)\\[-0.7cm]}
	\begin{itemize} \itemsep \isep
	  \item Opened and Analyzed a Cordless phone in a team of 3 members and written a 12-page Term paper with details of working of the phone.
	  \end{itemize}
\end{comment}
\end{itemize}

\resheading{\textbf{\large Scholastic Achievements}}
\begin{itemize}
  \item \textbf{All India Rank 61} of out 4.55 lakh students in IIT-JEE (Joint Entrance Examination) - 2010 \\[-0.7cm]
  
\item \textbf{All India Rank 3} of 18000 students in NEST (National Entrance Screening Test)-2010  \\[-0.7cm]%and \textbf{AIR 168} in AIEEE (All India Engineering Entrance Examination)- 2010. 

 % \item Qualified to appear for the Indian National Chemistry Olympiad (\textbf{INChO}) -2010 and Indian National Physics Olympiad (\textbf{INPhO}) -2010 (Top 300 selected nationwide based on performance in NSEP and NSEC)\\[-0.7cm]
  
  %\item Awarded \textbf{Certificate of Merit} by Central Board of Secondary Education (\textbf{CBSE}) for being among \textbf{top 0.1 \%} in `Science' and `Social Science' in All India Secondary School Examination - 2008 and Awarded '\textbf{Certificate of Excellence}' for securing highest aggregrate marks in the school \\[-0.7cm]
 \item Secured \textbf{AIR 5} in the  XXXIX National Mathematics Talent Competition (\textbf{NMTC})-2007 conducted by Association of Mathematics Teachers of India (\textbf{AMTI}).\\[-0.7cm]
  %\item Secured \textbf{AIR 15} in 10th National Science Olympiad (NSO) - 2007 and \textbf{State Rank 9} in 7th National Cyber Olympiad (NCO)- 2007  conducted by Science Olympiad Foundation(SOF).
  \\
\end{itemize}

\resheading{\textbf{\large Technical Skills}}
\begin{itemize}
  %\item Awarded Certificate of Participation in the Winter Workshop on Technical Skills conducted by \\ \ \  \ \ \ \  STUDe Club, IIT Bombay in January 2011. \\[-0.7cm]
  \item \textbf{Programming Languages}: C++,Java,Python,Ruby \  \ \ \ \ \ \ \  \textbf{Operating Systems}: Linux-Ubuntu, Windows\\[-0.6cm]
  \item \textbf{Tools}: Matlab, Mathematica, Scilab, Latex, Photoshop\ \ \ \  \textbf{Web designing}: HTML, CSS, Javascript\\[-0.6cm]
  \item \textbf{EE tools}: Spice, Verilog-HDL, Verilog-AMS , Microcontrollers
\end{itemize}

\begin{comment}
\resheading{\textbf{\large Technical Activities}}
\begin{itemize}
 \item Yahoo! HackU -2012: Built an android app and web interface, 'MapIt' which can be used to create customizable maps of localities with greater information\\[-0.7cm]
 \item Trackmania-2010 : Built a remote-controlled four-wheeled car (bot). \\[-0.7cm]
 	
 \item Line-follower competition-2011 :Built a line-following bot using IR sensors and Arduino.
\end{itemize}
\end{comment}
\begin{comment}
\resheading{\textbf{\large Extra Curricular Activities and Achievements}}
\begin{itemize}
  \item Participated in \textbf{Unnati}, the \textbf{NSS} (National Service Scheme) group of IIT Bombay.\\[-0.7cm]
	\begin{itemize}
	 \item Has been involved with the \textbf{GRA} (Group for Rural Activities) as part of curriculum in First year\\[-0.6cm]
	 \item Went to Village trips in Autumn 2010 and Spring 2011.\\[-0.6cm]
	 \item Continued as a voluntary member of the NSS Team in the subsequent year.\\[-0.6cm]
	  \end{itemize}
  \item Worked as `Organiser' in \textbf{Techfest-2011}, in the Lecture Series department. \\[-0.6cm]
  \item Participated in the Inter-hostel Hockey GC. 
\end{itemize}

\resheading{\textbf{\large Extra Courses currently taking (Autumn 2012)}}
  \begin{itemize}
  \item Data Structures and Algorithms
  \item Artificial Intelligence
  \item Foundations of VLSI CAD
\end{itemize}
\end{comment}
\resheading{\large{\textbf{Additional Data}}}
    \begin{itemize}
        \item \emph{Homepage} : \href{http://www.ee.iitb.ac.in/student/\~sksavant}{www.ee.iitb.ac.in/student/\textasciitilde{}sksavant} \\[-0.6cm]
        \item \emph{Github} : \href{http://www.github.com/sksavant}{www.github.com/sksavant}
    \end{itemize}

\end{document}
